\documentclass[12pt, a4paper]{article}
\usepackage[margin=1in]{geometry}
\usepackage[utf8x]{inputenc}
\usepackage{indentfirst} %indentace prvního odstavce
\usepackage{mathtools}
\usepackage{amsfonts}
\usepackage{amsmath}
\usepackage{amssymb}
\usepackage{graphicx}
\usepackage{enumitem}
\usepackage{subfig}
\usepackage{float}
\usepackage[czech]{babel}
\usepackage{mathdots}
\usepackage{slashbox}

\begin{document}
Rozkládáme číslo 3977 algoritmem ECM.
\begin{enumerate}
\item zvolíme $B \coloneqq 13$
\item zvolíme náhodně $a \coloneqq 3, d \coloneqq \gcd(3977, 4\cdot 3^3 + 27) = 1$
\item máme tedy $e_B = 2^3\cdot 3^2 \cdot 5 \cdot 7 \cdot 11 \cdot 13 = 360360$
\item $P = [0,1]$ a snažíme se spočítat $e_B \cdot P$
\item při výpočtu zjistíme, že $9\cdot P = [2706,1352]$ a bod $10 \cdot P = 9\cdot P + [0,1]$ nelze spočítat, jelikož $\gcd(2706-0,3977) = 41 \neq 1$.
\item Máme tedy faktor $41 | 3977 \implies 3977 = 41\cdot 97$
\end{enumerate}
\end{document}

