\documentclass[12pt, a4paper]{article}
\usepackage[margin=1in]{geometry}
\usepackage[utf8x]{inputenc}
\usepackage{indentfirst} %indentace prvního odstavce
\usepackage{mathtools}
\usepackage{amsfonts}
\usepackage{amsmath}
\usepackage{amssymb}
\usepackage{graphicx}
\usepackage{enumitem}
\usepackage{subfig}
\usepackage{float}
\usepackage[czech]{babel}
\usepackage{mathdots}
\usepackage{slashbox}

\begin{document}
Rozkládáme číslo 3977 algoritmem ECM.
\begin{enumerate}
\item zvolíme $B \coloneqq 13$
\item zvolíme náhodně $a \coloneqq 3, d \coloneqq \gcd(3977, 4\cdot 3^3 + 27) = 1$
\item máme tedy $e_B = 2^3\cdot 3^2 \cdot 5 \cdot 7 \cdot 11 \cdot 13 = 360360$
\item $P_0 = [0,1]$ a snažíme se spočítat $e_B \cdot P_0$ pomocí metody "binárního násobení" viz přednáška. Definujeme $Q = 0, P = P_0$ 
\item Číslo $e_B$ je dělitelné $2^3$ viz výše. Spočítáme tedy bod $P = 2^3 P_0 = 8P_0 = [70, 1353]$ a přičteme ho k $Q \implies Q = 8[0,1] = [70, 1353]$.
\item V proměnné $P$ je nyní hodnota $8[0,1] = [70, 1353]$. V dalším kroku chceme počítat bod $P = P+P = 16[0,1] = [70, 1353] + [70, 1353]$. Při výpočtu inverzu $(2 \cdot 1353)^{-1} = 2706^{-1} \mod 3977$ výpočet selže, jelikož inverz neexistuje, protože $\gcd(2706,3977) \neq 1$
\item spočteme tedy $\gcd(2706,3977) = 41$ a máme faktor $3977$. Druhý dopočítáme vydělením první faktorem.
\item Dostaneme $3977 = 41\cdot 97$
\end{enumerate}
\end{document}

