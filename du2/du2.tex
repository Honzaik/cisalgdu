\documentclass[12pt, a4paper]{article}
\usepackage[margin=1in]{geometry}
\usepackage[utf8x]{inputenc}
\usepackage{indentfirst} %indentace prvního odstavce
\usepackage{mathtools}
\usepackage{amsfonts}
\usepackage{amsmath}
\usepackage{amssymb}
\usepackage{graphicx}
\usepackage{enumitem}
\usepackage{subfig}
\usepackage{float}
\usepackage[czech]{babel}
\usepackage{mathdots}
\usepackage{slashbox}

\begin{document}
Rozkládáme číslo 3977 algoritmem Pollard p-1.
\begin{enumerate}
\item zvolíme $B \coloneqq 25$
\item zvolíme náhodně $a \coloneqq 127, d \coloneqq \gcd(3977, 127) = 1$
\item 
\begin{enumerate}
\item $p_1 \coloneqq 2$
\subitem $e \coloneqq \lfloor \log_2(25) \rfloor = 4$
\subitem $a \coloneqq 127^{2^4} \mod 3977 = 775$
\subitem $d \coloneqq \gcd(3977, 774) = 1$
\item $p_2 \coloneqq 3$
\subitem $e \coloneqq \lfloor \log_3(25) \rfloor = 2$
\subitem $a \coloneqq 775^{3^2} \mod 3977 = 3782$
\subitem $d \coloneqq \gcd(3977, 3781) = 1$
\item $p_2 \coloneqq 5$
\subitem $e \coloneqq \lfloor \log_5(25) \rfloor = 2$
\subitem $a \coloneqq 3782^{5^2} \mod 3977 = 2133$
\subitem $d \coloneqq \gcd(3977, 2132) = 41$
\end{enumerate}
\item return 41
\end{enumerate}
Zjistili jsme, že 41 je faktor čísla 3977. Platí $\frac{3977}{41} = 97$. Obě čísla jsou prvočísla. Tudíž $97, 41$ je prvočíselný rozklad 3977.

\end{document}

