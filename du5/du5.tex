\documentclass[12pt, a4paper]{article}
\usepackage[margin=1in]{geometry}
\usepackage[utf8x]{inputenc}
\usepackage{indentfirst} %indentace prvního odstavce
\usepackage{mathtools}
\usepackage{amsfonts}
\usepackage{amsmath}
\usepackage{amssymb}
\usepackage{graphicx}
\usepackage{enumitem}
\usepackage{subfig}
\usepackage{float}
\usepackage[czech]{babel}
\usepackage{mathdots}
\usepackage{slashbox}

\begin{document}
Chceme faktorizovat číslo $N=4399$, metodou CFRAC. Začneme tedy generovat relace.
\begin{enumerate}
\item bázi neřešíme
\item $S_0 = 1, p_{-1}=1, r_0 = 0, g = \lfloor 4366 \rfloor = 66, p_0 = g = 66$
\item $S_1 = 4399 - 66^2 = 43, R'_0 = 66, a_0 = 66, R'_1 = 132, n=1$
\begin{enumerate}
\item $n = 1$
\subitem $p_1 = 3 \cdot 66 + 1 = 199$
\subitem $R'_2 = 2\cdot 66 - 3 = 129$
\subitem $S_2 = 1 + 3\cdot(3-0) = 10 = 2 \cdot 5$
\subitem uložíme relaci $(199, 10)$
\item $n = 2$
\subitem $p_2 = 12 \cdot 199 + 66 = 2454$
\subitem $R'_3 = 132 - 9 = 123$
\subitem $S_3 = 43 + 12\cdot(9-3) = 115 = 5 \cdot 23$
\subitem uložíme relaci $(2454, -115)$
\item $n = 3$
\subitem $p_3 = 1 \cdot 2454 + 199 = 2653$
\subitem $R'_4 = 132 - 8 = 124$
\subitem $S_4 = 10 + 1\cdot(8-9) = 9 = 3^2$
\subitem uložíme relaci $(2653, 9)$
\end{enumerate}
\item máme relaci, která nám rovnou dává čtverec. Tím pádem spočítáme $\gcd(9+2653,4399)=83$
\end{enumerate}
Dopočítáme $4399/83 = 53$. Obě čísla jsou prvočísla, tedy prvočíselný rozklad $4399 = 53 * 83$.
\end{document}

