\documentclass[12pt, a4paper]{article}
\usepackage[margin=1in]{geometry}
\usepackage[utf8x]{inputenc}
\usepackage{indentfirst} %indentace prvního odstavce
\usepackage{mathtools}
\usepackage{amsfonts}
\usepackage{amsmath}
\usepackage{amssymb}
\usepackage{graphicx}
\usepackage{enumitem}
\usepackage{subfig}
\usepackage{float}
\usepackage[czech]{babel}
\usepackage{mathdots}
\usepackage{slashbox}

\begin{document}
Rozkládáme číslo 3763 algoritmem Pollard-Floyd.
\begin{enumerate}
\item $f(x) \coloneqq x^2 \mod 3763$
\item $x \coloneqq 1940$
\item Bude následovat tabulka výpočtů $x',x''$ a NSD$(3763, x'-x'')$ dokud NSD $\neq 1$ \\
\begin{table}[h]
\centering
\begin{tabular}{|l|l|l|}
\hline
$x'$ & $x''$ & NSD$(3763, x'-x'')$ \\ \hline
601 & 3717 & 1 \\ \hline
3717 & 3720 & 1 \\ \hline
2117 & 1934 & 1 \\ \hline
3720 & 463 & 1 \\ \hline
1850 & 3353 & 1 \\ \hline
1934 & 2505 & 1 \\ \hline
3698 & 1975 & 1 \\ \hline
463 & 2134 & 1 \\ \hline
3642 & 1710 & 1 \\ \hline
3353 & 2293 & 53 \\ \hline
\end{tabular}
\end{table}
\item $53 \neq 3763$, máme tedy vlastního dělitele čísla $3763 = 53 \cdot 71$. 
53 i 71 jsou prvočísla, takže tvoří prvočíselný rozklad 3763.
\end{enumerate}
\end{document}

